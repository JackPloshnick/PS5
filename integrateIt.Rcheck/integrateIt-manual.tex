\nonstopmode{}
\documentclass[a4paper]{book}
\usepackage[times,inconsolata,hyper]{Rd}
\usepackage{makeidx}
\usepackage[utf8,latin1]{inputenc}
% \usepackage{graphicx} % @USE GRAPHICX@
\makeindex{}
\begin{document}
\chapter*{}
\begin{center}
{\textbf{\huge Package `integrateIt'}}
\par\bigskip{\large \today}
\end{center}
\begin{description}
\raggedright{}
\item[Title]\AsIs{Integrate using trap and simpson}
\item[Version]\AsIs{0.1}
\item[Author]\AsIs{Jack Ploshnick}
\item[Maintainer]\AsIs{Jack Ploshnick }\email{j.ploshnick@wustl.edu}\AsIs{}
\item[Description]\AsIs{Approximates integral}
\item[Depends]\AsIs{R (>= 3.0.0), methods}
\item[License]\AsIs{GPL (>= 2)}
\item[Suggests]\AsIs{devtools}
\item[Collate]\AsIs{'Integrate function.R' 'Print method.R' 'Simpson Class.R'
'Simpson function.R' 'Trap Class.R' 'Trap function.R'}
\item[RoxygenNote]\AsIs{6.0.1}
\end{description}
\Rdcontents{\R{} topics documented:}
\inputencoding{utf8}
\HeaderA{integrateIt}{Integrates}{integrateIt}
\aliasA{addSquares,ANY-method}{integrateIt}{addSquares,ANY.Rdash.method}
%
\begin{Description}\relax
Integrates
\end{Description}
%
\begin{Usage}
\begin{verbatim}
integrateIt(x, y, start_end_values, test)
\end{verbatim}
\end{Usage}
%
\begin{Arguments}
\begin{ldescription}
\item[\code{x}] A numeric object

\item[\code{y}] A numeric object with the same dimensionality as \code{x}.

\item[\code{start\_end\_values}] A numeric

\item[\code{test}] a character
\end{ldescription}
\end{Arguments}
%
\begin{Value}
An object of class Trap or Simpson
\end{Value}
%
\begin{Note}\relax
Please Work
\end{Note}
%
\begin{Author}\relax
Jack Ploshnick
\end{Author}
%
\begin{SeeAlso}\relax
\code{\LinkA{subtractSquares}{subtractSquares}}
\end{SeeAlso}
\inputencoding{utf8}
\HeaderA{print,Trapezoid-method}{Prints}{print,Trapezoid.Rdash.method}
\aliasA{addSquares,ANY-method}{print,Trapezoid-method}{addSquares,ANY.Rdash.method}
%
\begin{Description}\relax
Prints
\end{Description}
%
\begin{Usage}
\begin{verbatim}
## S4 method for signature 'Trapezoid'
print(x)
\end{verbatim}
\end{Usage}
%
\begin{Arguments}
\begin{ldescription}
\item[\code{x}] A trap or simpson
\end{ldescription}
\end{Arguments}
%
\begin{Value}
A numeric
\end{Value}
%
\begin{Note}\relax
Please Work
\end{Note}
%
\begin{Author}\relax
Jack Ploshnick
\end{Author}
%
\begin{SeeAlso}\relax
\code{\LinkA{subtractSquares}{subtractSquares}}
\end{SeeAlso}
\inputencoding{utf8}
\HeaderA{Simpson-class}{Class of Simpson}{Simpson.Rdash.class}
\aliasA{addSquares,ANY-method}{Simpson-class}{addSquares,ANY.Rdash.method}
%
\begin{Description}\relax
Class of Simpson
\end{Description}
%
\begin{Arguments}
\begin{ldescription}
\item[\code{x}] A numeric object

\item[\code{y}] A numeric object with the same dimensionality as \code{x}.

\item[\code{estimate}] A numeric
\end{ldescription}
\end{Arguments}
%
\begin{Value}
creates Simpson Class
\end{Value}
%
\begin{Note}\relax
Please Work
\end{Note}
%
\begin{Author}\relax
Jack Ploshnick
\end{Author}
%
\begin{SeeAlso}\relax
\code{\LinkA{subtractSquares}{subtractSquares}}
\end{SeeAlso}
\inputencoding{utf8}
\HeaderA{simpsons}{Does simpson approximation}{simpsons}
\aliasA{addSquares,ANY-method}{simpsons}{addSquares,ANY.Rdash.method}
%
\begin{Description}\relax
Does simpson approximation
\end{Description}
%
\begin{Usage}
\begin{verbatim}
simpsons(x, y, StartToEnd)
\end{verbatim}
\end{Usage}
%
\begin{Arguments}
\begin{ldescription}
\item[\code{x}] A numeric object

\item[\code{y}] A numeric object with the same dimensionality as \code{x}.

\item[\code{StartToEnd}] A numeric
\end{ldescription}
\end{Arguments}
%
\begin{Value}
An object of class Trap or Simpson
\end{Value}
%
\begin{Note}\relax
Please Work
\end{Note}
%
\begin{Author}\relax
Jack Ploshnick
\end{Author}
%
\begin{SeeAlso}\relax
\code{\LinkA{subtractSquares}{subtractSquares}}
\end{SeeAlso}
\inputencoding{utf8}
\HeaderA{trap}{Does Trap approimation}{trap}
\aliasA{addSquares,ANY-method}{trap}{addSquares,ANY.Rdash.method}
%
\begin{Description}\relax
Does Trap approimation
\end{Description}
%
\begin{Usage}
\begin{verbatim}
trap(x, y, StartToEnd)
\end{verbatim}
\end{Usage}
%
\begin{Arguments}
\begin{ldescription}
\item[\code{x}] A numeric object

\item[\code{y}] A numeric object with the same dimensionality as \code{x}.

\item[\code{StartToEnd}] 
\end{ldescription}
\end{Arguments}
%
\begin{Value}
object of class trap
\end{Value}
%
\begin{Note}\relax
Please Work
\end{Note}
%
\begin{Author}\relax
Jack Ploshnick
\end{Author}
%
\begin{SeeAlso}\relax
\code{\LinkA{subtractSquares}{subtractSquares}}
\end{SeeAlso}
\inputencoding{utf8}
\HeaderA{Trapezoid-class}{Creates Trap class}{Trapezoid.Rdash.class}
\aliasA{addSquares,ANY-method}{Trapezoid-class}{addSquares,ANY.Rdash.method}
%
\begin{Description}\relax
Creates Trap class
\end{Description}
%
\begin{Arguments}
\begin{ldescription}
\item[\code{x}] A numeric object

\item[\code{y}] A numeric object with the same dimensionality as \code{x}.

\item[\code{estimate}] a numeric
\end{ldescription}
\end{Arguments}
%
\begin{Value}
makes Trap class
\end{Value}
%
\begin{Note}\relax
Please Work
\end{Note}
%
\begin{Author}\relax
Jack Ploshnick
\end{Author}
%
\begin{SeeAlso}\relax
\code{\LinkA{subtractSquares}{subtractSquares}}
\end{SeeAlso}
\printindex{}
\end{document}
